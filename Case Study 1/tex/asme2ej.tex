%%%%%%%%%%%%%%%%%%%%%%%%%%% asme2ej.tex %%%%%%%%%%%%%%%%%%%%%%%%%%%%%%%
% Template for producing ASME-format journal articles using LaTeX    %
% Written by   Harry H. Cheng, Professor and Director                %
%              Integration Engineering Laboratory                    %
%              Department of Mechanical and Aeronautical Engineering %
%              University of California                              %
%              Davis, CA 95616                                       %
%              Tel: (530) 752-5020 (office)                          %
%                   (530) 752-1028 (lab)                             %
%              Fax: (530) 752-4158                                   %
%              Email: hhcheng@ucdavis.edu                            %
%              WWW:   http://iel.ucdavis.edu/people/cheng.html       %
%              May 7, 1994                                           %
% Modified: February 16, 2001 by Harry H. Cheng                      %
% Modified: January  01, 2003 by Geoffrey R. Shiflett                %
% Butchered: October 15, 2014 by John Karasinski                %
% Use at your own risk, send complaints to /dev/null                 %
%%%%%%%%%%%%%%%%%%%%%%%%%%%%%%%%%%%%%%%%%%%%%%%%%%%%%%%%%%%%%%%%%%%%%%

%%% use twocolumn and 10pt options with the asme2ej format
\documentclass[twocolumn,10pt]{asme2ej}

\usepackage{epsfig} %% for loading postscript figures
\usepackage{listings}
\usepackage{amsmath}

%% The class has several options
%  onecolumn/twocolumn - format for one or two columns per page
%  10pt/11pt/12pt - use 10, 11, or 12 point font
%  oneside/twoside - format for oneside/twosided printing
%  final/draft - format for final/draft copy
%  cleanfoot - take out copyright info in footer leave page number
%  cleanhead - take out the conference banner on the title page
%  titlepage/notitlepage - put in titlepage or leave out titlepage
%  
%% The default is oneside, onecolumn, 10pt, final

\title{Case Study \# 1: 1D Transient Heat Diffusion}

%%% first author
\author{John Karasinski
    \affiliation{
	Graduate Student Researcher\\
	Center for Human/Robotics/Vehicle Integration and Performance\\
	Department of Mechanical Engineering\\
	University of California\\
	Davis, California 95616\\
    Email: karasinski@ucdavis.edu
    }	
}

\begin{document}
\maketitle    

%%%%%%%%%%%%%%%%%%%%%%%%%%%%%%%%%%%%%%%%%%%%%%%%%%%%%%%%%%%%%%%%%%%%%%
\section{Problem Description}

The problem of 1D unsteady heat diffusion in a slab of unit length with a zero initial temperature and both ends maintained at a unit temperature can be described by:
\begin{equation}
\frac{\partial T}{\partial t} = \frac{\partial^2T}{\partial x^2} \mbox{ subject to } \left\{ \begin{array}{lll}
        \mbox{$T(x, 0^-)= 0 $} & \mbox{for } &0 \leq x \leq 1 \\
        \mbox{$T(0, t) = T(1, t) = 1$} & \mbox{for } &t > 0 \end{array} \right.
\label{eq_DEF}
\end{equation}

and has a well-known analytical solution:
\begin{equation}
\begin{split}
T^*(x,t) = 1 - \sum\limits_{k=1}^\infty \frac{4}{(2k-1)\pi} & \sin[(2k-1)\pi x] \star \\
		    &  \exp[-(2k-1)^2\pi^2t].
\end{split}
\end{equation}

%%%%%%%%%%%%%%%%%%%%%%%%%%%%%%%%%%%%%%%%%%%%%%%%%%%%%%%%%%%%%%%%%%%%%%
\section{Solution Algorithms}

The Taylor-series (TS) method can be used on this equation to derive a finite difference approximation to the PDE. 

From the definition of the derivative, 
\begin{equation}
f'(x) \approx \frac{f(x + \epsilon) - f(x)}{\epsilon}
\end{equation}

and applying this to Eqn.~(\ref{eq_DEF}) yields:
\begin{equation}
\begin{split}
\frac{\partial T}{\partial t} = & \frac{T(x, t + \Delta t) - T(x,t)}{\Delta t} \\
				      = & \frac{T_{i}^{k+1} - T_{i}^{k}}{\Delta t}
\end{split}
\end{equation}

From the definition of the Taylor series,
\begin{equation}
f(x+\epsilon) = f(x) + \epsilon f'(x) + \frac{\epsilon^2}{2} f''(x) + ...
\end{equation}

Which, when applied to $T_{i}^{k+1}$  and $T_{i}^{k}$ gives:
\begin{equation}
T_{i+1} = T_i + \Delta x \frac{\partial T_i}{\partial x} + \frac{\Delta x^2}{2} \frac{\partial^2 T_i}{\partial x^2} + \mathcal{O}(\Delta x^3)
\end{equation}
and
\begin{equation}
T_{i-1} = T_i - \Delta x \frac{\partial T_i}{\partial x} + \frac{\Delta x^2}{2} \frac{\partial^2 T_i}{\partial x^2} - \mathcal{O}(\Delta x^3)
\end{equation}
which, when combined, yields:
\begin{equation}
T_{i+1} + T_{i-1} = 2T_i + \Delta x^2 \frac{\partial^2 T_i}{\partial x^2} + \mathcal{O}(\Delta x^4)
\end{equation}

This problem can be solved numerically using both Forward-Time, Centered-Space (FTCS) explicit and implicit schemes.
%%%%%%%%%%%%%%%%%%%%%%%%%%%%%%%%%%%%%%%%%%%%%%%%%%%%%%%%%%%%%%%%%%%%%%
\section{Results}

A Python script was used to obtain results for a 21 point mesh (N=21), and the Root Mean Square error,
\begin{equation}
RMS = dicks
\end{equation}
was obtained for $s(=\Delta t/ \Delta x^2)$ = 1/6, 0.25, 0.5, and 0.75, at $t$ = 0.03, 0.06, and 0.09.

%%%%%%%%%%%%%%%%%%%%%%%%%%%%%%%%%%%%%%%%%%%%%%%%%%%%%%%%%%%%%%%%%%%%%%
%%%%%%%%%%%%%%% begin table   %%%%%%%%%%%%%%%%%%%%%%%%%%
\begin{table}[t]
\caption{Figure and table captions do not end with a period}
\begin{center}
\label{table_ASME}
\begin{tabular}{c l l}
& & \\ % put some space after the caption
\hline
Example & Time & Cost \\
\hline
1 & 12.5 & \$1,000 \\
2 & 24 & \$2,000 \\
\hline
\end{tabular}
\end{center}
\end{table}
%%%%%%%%%%%%%%%% end table %%%%%%%%%%%%%%%%%%% 
%%%%%%%%%%%%%%%%%%%%%%%%%%%%%%%%%%%%%%%%%%%%%%%%%%%%%%%%%%%%%%%%%%%%%%

All tables should be numbered consecutively  and centered above the table as shown in Table~\ref{table_ASME}. The body of the table should be no smaller than 7 pt.  There should be a minimum two line spaces between tables and text.

%%%%%%%%%%%%%%%%%%%%%%%%%%%%%%%%%%%%%%%%%%%%%%%%%%%%%%%%%%%%%%%%%%%%%%
\section{Discussions}
Shankle chicken tail, fatback short ribs meatball pancetta ball tip sirloin short loin. Pork tongue pork belly pork loin beef ribs. Shank turkey pork belly pork loin ham hock ball tip leberkas meatloaf chuck ground round filet mignon kielbasa sirloin turducken tri-tip. Pancetta brisket sirloin beef ribs spare ribs, swine bacon ham hock. Ham kielbasa corned beef turkey turducken. Kevin biltong pork, tenderloin chuck pig ball tip filet mignon.

%%%%%%%%%%%%%%%%%%%%%%%%%%%%%%%%%%%%%%%%%%%%%%%%%%%%%%%%%%%%%%%%%%%%%%
% The bibliography is stored in an external database file
% in the BibTeX format (file_name.bib).  The bibliography is
% created by the following command and it will appear in this
% position in the document. You may, of course, create your
% own bibliography by using thebibliography environment as in
%
% \begin{thebibliography}{12}
% ...
% \bibitem{itemreference} D. E. Knudsen.
% {\em 1966 World Bnus Almanac.}
% {Permafrost Press, Novosibirsk.}
% ...
% \end{thebibliography}

% Here's where you specify the bibliography style file.
% The full file name for the bibliography style file 
% used for an ASME paper is asmems4.bst.
%\bibliographystyle{asmems4}

% Here's where you specify the bibliography database file.
% The full file name of the bibliography database for this
% article is asme2e.bib. The name for your database is up
% to you.
%\bibliography{asme2e}

%%%%%%%%%%%%%%%%%%%%%%%%%%%%%%%%%%%%%%%%%%%%%%%%%%%%%%%%%%%%%%%%%%%%%%
%\appendix       %%% starting appendix
%\section*{Appendix A: Head of First Appendix}
%Avoid Appendices if possible.

%%%%%%%%%%%%%%%%%%%%%%%%%%%%%%%%%%%%%%%%%%%%%%%%%%%%%%%%%%%%%%%%%%%%%%
%\section*{Appendix B: Head of Second Appendix}
%\subsection*{Subsection head in appendix}
%The equation counter is not reset in an appendix and the numbers will
%follow one continual sequence from the beginning of the article to the very end as shown in the following example.
%\begin{equation}
%a = b + c.
%\end{equation}

\end{document}
