%%%%%%%%%%%%%%%%%%%%%%%%%%% asme2ej.tex %%%%%%%%%%%%%%%%%%%%%%%%%%%%%%%
% Template for producing ASME-format journal articles using LaTeX    %
% Written by   Harry H. Cheng, Professor and Director                %
%              Integration Engineering Laboratory                    %
%              Department of Mechanical and Aeronautical Engineering %
%              University of California                              %
%              Davis, CA 95616                                       %
%              Tel: (530) 752-5020 (office)                          %
%                   (530) 752-1028 (lab)                             %
%              Fax: (530) 752-4158                                   %
%              Email: hhcheng@ucdavis.edu                            %
%              WWW:   http://iel.ucdavis.edu/people/cheng.html       %
%              May 7, 1994                                           %
% Modified: February 16, 2001 by Harry H. Cheng                      %
% Modified: January  01, 2003 by Geoffrey R. Shiflett                %
% Butchered: October 15, 2014 by John Karasinski                     %
% Use at your own risk, send complaints to /dev/null                 %
%%%%%%%%%%%%%%%%%%%%%%%%%%%%%%%%%%%%%%%%%%%%%%%%%%%%%%%%%%%%%%%%%%%%%%

%%% use twocolumn and 10pt options with the asme2ej format
\documentclass[twocolumn,10pt]{asme2ej}

\usepackage{epsfig} %% for loading postscript figures
\usepackage{amsmath}
\usepackage{graphicx}
\usepackage{grffile}
\usepackage{pdfpages}
\usepackage{algpseudocode}

% Default fixed font does not support bold face
\DeclareFixedFont{\ttb}{T1}{txtt}{bx}{n}{12} % for bold
\DeclareFixedFont{\ttm}{T1}{txtt}{m}{n}{12}  % for normal

% Custom colors
\usepackage{color}
\usepackage{listings}
\usepackage{framed}
\usepackage{caption}
\usepackage{bm}
\captionsetup[lstlisting]{font={small,tt}}

\definecolor{mygreen}{rgb}{0,0.6,0}
\definecolor{mygray}{rgb}{0.5,0.5,0.5}
\definecolor{mymauve}{rgb}{0.58,0,0.82}

\lstset{ %
  backgroundcolor=\color{white},   % choose the background color; you must add \usepackage{color} or \usepackage{xcolor}
  basicstyle=\ttfamily\footnotesize, % the size of the fonts that are used for the code
  breakatwhitespace=false,         % sets if automatic breaks should only happen at whitespace
  % breaklines=true,                 % sets automatic line breaking
  captionpos=b,                    % sets the caption-position to bottom
  commentstyle=\color{mygreen},    % comment style
  deletekeywords={...},            % if you want to delete keywords from the given language
  escapeinside={\%*}{*)},          % if you want to add LaTeX within your code
  extendedchars=true,              % lets you use non-ASCII characters; for 8-bits encodings only, does not work with UTF-8
  frame=single,                    % adds a frame around the code
  keepspaces=true,                 % keeps spaces in text, useful for keeping indentation of code (possibly needs columns=flexible)
  columns=flexible,
  keywordstyle=\color{blue},       % keyword style
  language=Python,                 % the language of the code
  morekeywords={*,...},            % if you want to add more keywords to the set
  numbers=left,                    % where to put the line-numbers; possible values are (none, left, right)
  numbersep=5pt,                   % how far the line-numbers are from the code
  numberstyle=\tiny\color{mygray}, % the style that is used for the line-numbers
  rulecolor=\color{black},         % if not set, the frame-color may be changed on line-breaks within not-black text (e.g. comments (green here))
  showspaces=false,                % show spaces everywhere adding particular underscores; it overrides 'showstringspaces'
  showstringspaces=false,          % underline spaces within strings only
  showtabs=false,                  % show tabs within strings adding particular underscores
  stepnumber=1,                    % the step between two line-numbers. If it's 1, each line will be numbered
  stringstyle=\color{mymauve},     % string literal style
  tabsize=4,                       % sets default tabsize to 2 spaces
}

\title{Case Study \# 5: Two-Species Diffusion-Diurnal Kinetics}

\author{John Karasinski
    \affiliation{
  Graduate Student Researcher\\
  Center for Human/Robotics/Vehicle Integration and Performance\\
  Department of Mechanical and Aerospace Engineering\\
  University of California\\
  Davis, California 95616\\
    Email: karasinski@ucdavis.edu
    }
}

\begin{document}
\maketitle

%%%%%%%%%%%%%%%%%%%%%%%%%%%%%%%%%%%%%%%%%%%%%%%%%%%%%%%%%%%%%%%%%%%%%%
\section{Problem Description}

Chang et al.~\cite{chang1974simulation, byrne1987stiff} have proposed approximate models to describe the chemical kinetics and transport phenomena associated with the dissociation of oxygen (O$_2$) into ozone (O$_3$) and monatomic oxygen (O) in the upper atmosphere. A one-dimensional version of such a model is considered here. The ambient oxygen concentration, $c_3$, is constant, while the concentrations of the two minor species, O and O$_3$, are $c_1(z, t)$ and $c_2(z, t)$, where $z$ is the elevation above the earth's surface in km (here $30 \leq z \leq 50$) and $t$ is time in seconds. Their transport is modeled using a reaction-diffusion equation,
\begin{equation}
\label{reaction_diffusion}
\frac{\partial c_i}{\partial t} = \frac{\partial}{\partial z} \left[K(z) \frac{\partial c_i}{dz} \right] + R_i (\vec{c}, t) \qquad i = 1, 2, 3.
\end{equation}
The diffusive term is meant to represent the turbulent vertical transport with
\begin{equation}
K(z) = 10^{-8} \cdot exp(z/5) \qquad [\mbox{km}/\mbox{s}],
\end{equation}
 and the chemistry is described using the Chapman mechanism~\cite{byrne1987stiff}. The reaction rates, $R_i(c, t)$, are given by
\begin{equation}
\begin{split}
R_1(c_1, c_2, t) = & -k_1 c_1 c_3 - k_2 c_1 c_2 + 2 k_3(t) c_3 + k_4(t) c_2 \\
R_2(c_1, c_2, t) = &  k_1 c_1 c_3 - k_2 c_1 c_2 - k_4(t) c_2
\end{split}
\end{equation}
with,
\begin{equation*}
\begin{split}
k_1 = & 1.63 x 10^{−16} \\
k_2 = & 4.66 x 10^{−16} \\
k_l = & \mbox{exp} [−a_l/ \mbox{sin}~\omega~t]~\mbox{if sin}~\omega~t~>~0,~\mbox{else}~0~(l = 3, 4)
\end{split}
\end{equation*}
and with $a_3$ = 22.62, $a_4$ = 7.601, and $\omega = \pi/43200$. This system is subject to the initial conditions,
\begin{equation}
\begin{split}
c_1(z, 0) = & 10^6    \cdot \gamma(z)     \\
c_2(z, 0) = & 10^{12} \cdot \gamma(z),
\end{split}
\end{equation}
where
\begin{equation}
\gamma(z) = 1 - \left(\frac{z-40}{10} \right)^2 + \frac{1}{2} \left(\frac{z-40}{10} \right)^4,
\end{equation}
and a boundary condition of no flux at the top and bottom of the vertical atmospheric layer considered.

%%%%%%%%%%%%%%%%%%%%%%%%%%%%%%%%%%%%%%%%%%%%%%%%%%%%%%%%%%%%%%%%%%%%%%
\section{Numerical Solution Approach}

To generate a system of ordinary differential equations, all the spatial derivatives in Equation~\ref{reaction_diffusion} are replaced with centered finite differences. For the base case considered, the domain is discretized into $M = 50$ partitions, $\Delta z = 20/M$, and $z_j = 30 + j(\Delta z)$ for $0 \leq j \leq M$.

The function $c^i(z_j,t)$ can then be approximated as
\begin{equation}
\begin{split}
\dot c^i_j = (\Delta z)^{-2} [& K_{j+1/2} c^i_{j+1} - (K_{j+1/2} +  K_{j-1/2}) c^i_{j} + \\
& K_{j-1/2} c^i_{j-1}] + R^i(\boldsymbol{c}, t),
\end{split}
\end{equation}

\noindent where $K_{j \pm 1/2} = K(30+[j \pm 1/2] \Delta z)$. A system of $2M$ ODEs is then specified by setting $\boldsymbol{y}(t)=[c^1_1(t), c^2_1(t), c^1_2(t), c^2_2(t), \ldots, c^1_M(t),c^2_M(t)]^T$, with boundary conditions $c^i_0 = c^i_2$ and $c^i_{M-1} = c^i_{M+1}$. The two parabolic PDEs are then reduced to a system of 2M ODEs of the form $\dot{\boldsymbol{y}} = \boldsymbol{f} (t, \boldsymbol{y})$.

Two solvers are called from scipy version 0.11.0 to solve this system of ODEs. A stiff solver, ``vode", is an implicit method based on the backward differentiation formulas, and a non-stiff solver, ``dopri5", is an explicit Runge-Kutta method of order (4,5) are used.

%%%%%%%%%%%%%%%%%%%%%%%%%%%%%%%%%%%%%%%%%%%%%%%%%%%%%%%%%%%%%%%%%%%%%%
\section{Results Discussion}

\subsection{Comparison to Published Results}
\begin{figure}[tb]
\begin{center}
\includegraphics[width=0.5\textwidth]{../code/figures/bdf 86400.0 50 c1.pdf}
\caption{$c_1$ vs. z at $t = $ 0, 2, 4, 6, 7, and 9 hours}
\label{c1_plot}
\end{center}
\end{figure}

\begin{figure}[tb]
\begin{center}
\includegraphics[width=0.5\textwidth]{../code/figures/bdf 86400.0 50 c2.pdf}
\caption{$c_2$ vs. z at $t = $ 0, 2, 4, 6, 7, 9, 12, 18, and 24 hours}
\label{c2_plot}
\end{center}
\end{figure}

\begin{figure}[tb]
\begin{center}
\includegraphics[width=0.5\textwidth]{../code/figures/bdf 864000.0 50 time.pdf}
\caption{$c_1$ and $c_2$ vs. time (from 0 to 10 days) at $z = 40$ km, $c_2$ is scaled by 1E-4}
\label{40km_plot}
\end{center}
\end{figure}

Both methods required an absolute tolerance of 1E-1 and a relative tolerance 1E-3 for convergence. The results found for both the bdf and dopri5 solvers are effectively the save when plotted on a logarithmic scale. The results from the dopri5 solver can be seen in Figures~\ref{c1_plot},~\ref{c2_plot}, and~\ref{40km_plot}, and compare well with published results~\cite{chang1974simulation, byrne1987stiff}.

In particular the ordering of the $c_1$ concentrations are the same from $t = $ 0 to $t = $ 9 hours. From highest to lowest concentrations, the ordering is $t = $ 6, 7, 4, 9, 0, and 2 hours. The ordering of the $c_2$ concentrations also agree over the time range of $t = $ 0 to $t = $ 24 hours. From highest to lowest concentrations at the upper boundary of 50 km, the ordering is $t = $ 24, 18, 12, 9, 7, 6, 4, 2, and 0 hours. Additionally, the concentrations of both species were investigated at a height of 40 km as a function of time. A peak is seen in the concentration of $c_1$ during each day. This peak grows slightly as time elapses. The concentration of $c_2$ acts as a step function, stepping up each day at approximately the same time as the $c_1$ concentration peaks.

The results for concentrations of $c_1$ and $c_2$ as a function of altitude, and the results for the concentrations of $c_1$ and $c_2$ at 40 km as a function of time agree with the results in both reference papers.

\subsection{Difference in convergence time}
\begin{table}[tb]
\begin{center}
\begin{tabular}{| l | r r |}
\hline
Solver & 24 hours & 10 days \\
\hline
dopri5 & 1025     & 10069   \\
bdf    & 1318     & 13237   \\
\hline
\end{tabular}
\caption{Wall clock time, in seconds, for each solver to solve to $t=24$ hours and $t=10$ days}
\label{run_time}
\end{center}
\end{table}

For stiff problems such as this, stiff solvers should take longer to convergence than non-stiff solvers. Table~\ref{run_time} shows the wall clock time, in seconds, to converge for the 24 hour and 10 day cases. The bdf solver takes about 31\% longer to converge than the dopri5 solver, suggesting that this problem is indeed stiff.

\subsection{Sensitivity to Mesh Density}

The effect of varying the mesh size was also investigated. The solution was found for the concentrations of both species for both solvers at $t = $ 4 hours using meshes resultant from M = 5, 10, 25, 50, 75, 100, and 200. The solution with the mesh resulting from M = 200 was taken to be effectively equivalent to the analytical solution, and was used as the basis for comparison between the solutions found with the other meshes.

The Root Mean Square error,
\begin{equation}
RMSE = \sqrt{\frac{1}{N}\sum\limits_{i=1}^N[c_i - c^*_i]^2},
\end{equation}
and the Normalized Root Mean Square error,
\begin{equation}
NRMS = \dfrac{RMSE}{max(c^*_i)-min(c^*_i)},
\end{equation}
can be calculated. Here $c_i$ is the result for the the concentration of each species $i$ at each point on the 1-D domain generated from each value for M, $c^*_i$ is the solution from the M = 200 simulation, and $N$ is the number of points on the 1-D domain. The $NRMS$ for each case is expressed as a percentage, where lower values indicate a result closer to the analytic solution. For the complete NRMS results for each solver and mesh, see Tables~\ref{dopri5_nrms_table} and~\ref{bdf_nrms_table}.

These results show that the difference between the meshes decreases rapidly for larger M. At M = 50, the NRMS is around 1.8E-3 for both concentrations using both solvers. This relatively low value indicates that using a mesh from M = 50 leads to relatively good results with the benefit of a reduced solver time. A plot of these results can be seen as Figure~\ref{NRMS_plot}.


\begin{table}[tb]
\begin{center}
\begin{tabular}{| r | r r |}
\hline
M & $c_{1}$ & $c_{2}$ \\
\hline
  5 & 1.628E-2 & 1.648E-2 \\
 10 & 9.766E-3 & 9.829E-3 \\
 25 & 4.178E-3 & 4.110E-3 \\
 50 & 1.879E-3 & 1.811E-3 \\
 75 & 1.031E-3 & 9.806E-4 \\
100 & 6.923E-4 & 6.098E-4 \\
\hline
\end{tabular}
\caption{NRMS of results at $t=4$ hours compared to results at M = 200 for the dopri5 solver}
\label{dopri5_nrms_table}
\end{center}
\end{table}

\begin{table}[tb]
\begin{center}
\begin{tabular}{| r | r r |}
\hline
M & $c_{1}$ & $c_{2}$ \\
\hline
  5 & 1.654E-2 & 1.648E-2 \\
 10 & 9.796E-3 & 9.829E-3 \\
 25 & 4.091E-3 & 4.110E-3 \\
 50 & 1.802E-3 & 1.811E-3 \\
 75 & 9.436E-4 & 9.806E-4 \\
100 & 5.982E-4 & 6.098E-4 \\
\hline
\end{tabular}
\caption{NRMS of results at $t=4$ hours compared to results at M = 200 for the bdf solver}
\label{bdf_nrms_table}
\end{center}
\end{table}

\begin{figure}[tb]
\begin{center}
\includegraphics[width=0.5\textwidth]{../code/figures/[5, 10, 25, 50, 75, 100, 200].pdf}
\caption{NRMS for M = 5, 10, 25, 50, 75, and 100 compared against M = 200 for both solvers and concentrations}
\label{NRMS_plot}
\end{center}
\end{figure}

%%%%%%%%%%%%%%%%%%%%%%%%%%%%%%%%%%%%%%%%%%%%%%%%%%%%%%%%%%%%%%%%%%%%%%
\section{Conclusion}
A previously developed model to describe the chemical kinetics and transport phenomena associated with the dissociation of oxygen in the upper atmosphere was investigated and the results were compared to previously published results. A numerical solution of the system was discretized using the method of lines and central differencing for the diffusion term on a uniform mesh with M nodes.

A Python script was created and scipy was used to call both a stiff and non-stiff ODE solver to investigate the concentrations of two species of oxygen as a function of altitude and time. This script can be seen as Appendix~1. The results of these two solvers both compared well to previously published results, though the stiff solver took approximately 31\% longer to converge. A mesh sensitivity study was also performed, and confirmed that the base mesh resulting from M = 50 was adequate for the values investigated here.

%%%%%%%%%%%%%%%%%%%%%%%%%%%%%%%%%%%%%%%%%%%%%%%%%%%%%%%%%%%%%%%%%%%%%%
\nocite{*}
\bibliographystyle{asmems4}
\bibliography{asme2e}

%%%%%%%%%%%%%%%%%%%%%%%%%%%%%%%%%%%%%%%%%%%%%%%%%%%%%%%%%%%%%%%%%%%%%%
\clearpage
\onecolumn
\appendix       %%% starting appendix
\section*{Appendix A: Python Code}

\lstinputlisting[caption=Code to create solutions, language=Python]{../code/CaseStudy5.py}
\lstinputlisting[caption=Code to generate pretty plots, language=Python]{../code/PrettyPlots.py}

%%%%%%%%%%%%%%%%%%%%%%%%%%%%%%%%%%%%%%%%%%%%%%%%%%%%%%%%%%%%%%%%%%%%%%
\end{document}
